\section{Related Works}
\label{sec:related}

Anderson et al. \cite{Anderson:2016, Anderson:2017} present a collection of supervised machine learning algorithms for ETA. The authors highlighted two primary reasons why traditional pattern matching approaches cannot be applied to ETA, namely, inaccurate ground truth and a highly non-stationary data distribution. To address the aforementioned problems, the authors develop supervised machine learning models that leverage a unique and diverse set of network flow data features, including 1) TLS handshake meta-features, 2) DNS contextual flows linked to the encrypted flow, 3) HTTP headers of HTTP contextual flows from the same source IP address within a 5-minute window, and 4) META feature. Experiment results show that by incorporating the contextual information into a supervised learning system, the ETA system can achieve 0.00\% false discovery rate and high detection accuracy. 

\texttt{Joy} \cite{joy} is an open-sourced software initiated by the same authors Anderson et al. \cite{Anderson:2016, Anderson:2017}. Though lacking of the network traffic data for reference, the published version of Joy is capable of generating the aforementioned data features. In this paper, we leverage Joy as the feature extraction stage of the \texttt{ACETA} pipeline, and extend the only model published in Joy, namely logistic regression, to a suite of machine learning/deep learning models. More importantly, we implemented an accelerated version of the model suite with Intel DAAL and OpenVINO, and conduct extensive performance study over a public dataset \texttt{CICAndMal2017} \cite{cicandmal2017}.